\chapter{Non-functional requirements}
\label{nonfunctional}

This whole chapter is Duncan-suitable :)
\duncan

\section{Network layer}
\label{nonfunctional:network}

This report is not intended as a comprehensive discussion of the network layer;
see \cite{network-spec} instead. However, in order to understand
some of the design decisions in the consensus layer we need to understand some
of the requirements imposed on it by the network layer.

TODOs:

\begin{itemize}
\item Highlight relevant aspects of the design of the network layer
\item Discuss requirements this imposes on the consensus layer
Primary example: Forecasting.
\item How do we keep the overlap between network and consensus as small
as possible? Network protocols do not involve consensus protocols
(chain sync client is not dependent on chain selection). Chain sync
client + "pre chain selection" + block download logic keeps things isolated.
\item Why do we even want to validate headers ahead of time? (Thread model etc.)
(Section for Duncan?).
Section with a sketch on an analysis of the amortised cost for attackers versus
our own costs to defend against it ("budget for work" that grows and shrinks
as you interact with a node).
\end{itemize}

\subsection{Header/Body Split (aka: Header submission)}
\label{nonfunctional:network:headerbody}

Discuss the chain fragments that we store per upstream node.
Discuss why we want to validate headers here -- without a full ledger state
(necessarily so, since no block bodies -- can't update ledger state): to prevent
DoS attacks.
(\cref{ledger:forecasting} contains a discussion of this from the point of view of
the ledger).
Forward reference to the chain sync client (\cref{chainsyncclient}).
Discuss why it's useful if the chain sync client can race ahead  for
\emph{performance} (why it's required for chain selection is the discussed in
\cref{forecast:ledgerview}).

See also section on avoiding the stability window
(\cref{low-density:pre-genesis}).

\subsection{Block submission}
\label{nonfunctional:network:blocksubmission}

Forward reference to \cref{servers:blockfetch}.

\subsection{Transaction submission}
\label{nonfunctional:network:txsubmission}

Mention that these are defined entirely network side, no consensus involvement
(just an abstraction over the mempool).

\section{Security "cost" concerns}

TODO: Look through the code and git history to find instances of where we
one way but not the other because it would give an attacker an easy way to
make it do lots of work (where were many such instances).

Fragile. Future work: how might be make this less brittle?
Or indeed, how might we test this?

Counter-examples (things we don't want to do)

\begin{itemize}
\item Parallel validation of an entire epoch of data (say, crypto only).
You might do a lot of work before realising that that work was not needed because
of an invalid block in the middle.
\end{itemize}

Future work: opportunities for parallelism that we don't yet exploit
(important example: script evaluation in Goguen).

\section{Hard time constraints}

Must produce a block on time, get it to the next slot leader

Bad counter-example: reward calculation in the Shelley ledger bad
(give examples of why).

\section{Predictable resource requirements}
\label{nonfunctional:best-is-worst}

make best == worst

(not \emph{just} a security concern: a concern even if every node honest)
