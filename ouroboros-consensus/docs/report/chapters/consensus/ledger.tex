\chapter{Interface to the ledger}
\label{ledger}

\section{Abstract interface}
\label{ledger:api}

In \cref{overview:ledger} we identified three responsibilities for the ledger
layer:
%
\begin{itemize}
\item ``Ticking'' the ledger state, applying any time related changes
(\cref{ledger:api:IsLedger}). This is independent from blocks, both at the value
level (we don't need a block in order to tick) and at the type level.
\item Applying and verifying blocks (\cref{ledger:api:ApplyBlock}). This
obviously connects a ledger and a block type, but we try to avoid to talk about
\emph{the} ledger corresponding to a block, in order to improve
compositionality; we will see examples of where this comes in useful in the
definition of the extended ledger state (\cref{storage:extledgerstate}) and the
ledger database (\cref{ledgerdb}).
\item Projecting out the ledger view (\cref{ledger:api:LedgerSupportsProtocol}),
connecting a ledger to a consensus protocol.
\end{itemize}
%
We will discuss these responsibilities one by one.

\subsection{Independent definitions}
\label{ledger:api:IsLedger}

We will start with ledger API that can be defined independent of a choice of
block or a choice of consensus protocol.

\subsubsection{Configuration}

Like the other abstractions in the consensus layer, the ledger defines its own
type of required static configuration
%
\begin{lstlisting}
type family LedgerCfg l :: Type
\end{lstlisting}

\subsubsection{Tip}

We require that any ledger can report its tip as a \lstinline!Point!. A
\lstinline!Point! is either genesis (no blocks have been applied yet) or a pair
of a hash and slot number; it is parametric over $l$ in order to allow
different ledgers to use different hash types.
%
\begin{lstlisting}
class GetTip l where
  getTip :: l -> Point l
\end{lstlisting}

\subsubsection{Ticking}

We can now define the \lstinline!IsLedger! class as
%
\begin{lstlisting}
class (GetTip l, GetTip (Ticked l), ..) => IsLedger l where
  type family LedgerErr l :: Type
  applyChainTick :: LedgerCfg l -> SlotNo -> l -> Ticked l
\end{lstlisting}

The type of \lstinline!applyChainTick! is similar to the type of
\lstinline!tickChainDepState! we saw in \cref{consensus:class:state}.
Examples of the time-based changes in the ledger state include activating
delegation certificates in the Byron ledger, or paying out staking rewards
in the Shelley ledger.

Ticking is not allowed to fail (it cannot return an error). Consider what it
would mean if it \emph{could} fail: it would mean that a previous block was
accepted as valid, but set up the ledger state so that no matter what would
happen next, as soon as a particular moment in time is reached, the ledger would
fail to advance any further. Obviously, such a situation cannot be permitted to
arise (the block should have been rejected as invalid).

Note that ticking does not change the tip of the ledger: no blocks have been
applied (yet). This means that we should have

\begin{equation}
  \mathtt{getTip} \; l
= \mathtt{getTip} \; (\mathtt{applyChainTick}_\mathit{cfg} \; s \; l)
\end{equation}

\subsubsection{Ledger errors}

The inclusion of \lstinline!LedgerErr! in \lstinline!IsLedger! is perhaps
somewhat surprising. \lstinline!LedgerErr! is the type of errors that can arise
when applying blocks to the ledger, but block application is not yet defined
here. Nonetheless, a ledger can only be applied to a \emph{single} type of
block, and consequently can only have a \emph{single} type of error; the only
reason block application is defined separately is that a single type of
\emph{block} can be used with multiple ledgers (in other words, this is a
1-to-many relationship).\footnote{Defining \lstinline!LedgerErr! in
\lstinline!ApplyBlock! (\cref{ledger:api:ApplyBlock}) would result in ambiguous
types, since it would not refer to the \lstinline!blk! type variable of that
class.}

\subsection{Applying blocks}
\label{ledger:api:ApplyBlock}

If \lstinline!applyChainTick! was analogous to \lstinline!tickChainDepState!,
then \lstinline!applyLedgerBlock! and \lstinline!reapplyLedgerBlock! are
analogous to \lstinline!updateChainDepState! and
\lstinline!reupdateChainDepState!, respectively
(\cref{consensus:class:state}): apply a block to an already ticked
ledger state:
%
\begin{lstlisting}
class (IsLedger l, ..) => ApplyBlock l blk where
  applyLedgerBlock ::
    LedgerCfg l -> blk -> Ticked l -> Except (LedgerErr l) l
  reapplyLedgerBlock ::
    LedgerCfg l -> blk -> Ticked l -> l
\end{lstlisting}
%
The discussion of the difference between, and motivation for, the distinction
between application and reapplication in \cref{consensus:class:state}
about the consensus protocol state applies here equally.

\subsection{Linking a block to its ledger}

We mentioned at the start of \cref{ledger:api} that a single block can be used
with multiple ledgers. Nonetheless, there is one ``canonical'' ledger for each
block; for example, the Shelley block is associated with the Shelley ledger,
even if it can also be applied to the extended ledger state or the ledger
DB. We express this through a data family linking a block to its ``canonical
ledger state'':
%
\begin{lstlisting}
data family LedgerState blk :: Type
\end{lstlisting}
%
and then require that it must be possible to apply a block to its associated
ledger state
%
\begin{lstlisting}
class ApplyBlock (LedgerState blk) blk => UpdateLedger blk
\end{lstlisting}
%
(this is an otherwise empty class). For convenience, we then also introduce
some shorthand:
%
\begin{lstlisting}
type LedgerConfig      blk = LedgerCfg (LedgerState blk)
type LedgerError       blk = LedgerErr (LedgerState blk)
type TickedLedgerState blk = Ticked    (LedgerState blk)
\end{lstlisting}

\subsection{Projecting out the ledger view}
\label{ledger:api:LedgerSupportsProtocol}

In \cref{overview:ledger} we mentioned that a consensus protocol may require
some information from the ledger, and in \cref{consensus:class:ledgerview} we
saw that this is modelled as the \lstinline!LedgerView! type family in the
\lstinline!ConsensusProtocol! class. A ledger and a consensus protocol are
linked through the block type (indeed, apart from the fundamental concepts we
have discussed so far, most of consensus is parameterised over blocks, not
ledgers or consensus protocols). Recall from \cref{BlockSupportsProtocol} that
the \lstinline!BlockProtocol! type family defines for each block what the
corresponding consensus protocol is; we can use this to define the projection of
the ledger view (defined by the consensus protocol) from the ledger state as
follows:
%
\begin{lstlisting}
class (..) => LedgerSupportsProtocol blk where
  protocolLedgerView ::
       LedgerConfig blk
    -> Ticked (LedgerState blk)
    -> Ticked (LedgerView (BlockProtocol blk))

  ledgerViewForecastAt ::
       LedgerConfig blk
    -> LedgerState blk
    -> Forecast (LedgerView (BlockProtocol blk))
\end{lstlisting}
%
The first method extracts the ledger view out of an already ticked ledger state;
think of it as the ``current'' ledger view. Forecasting deserves a more detailed
discussion and will be the topic of the next section.

\section{Forecasting}
\label{ledger:forecasting}

\subsection{Introduction}

In \cref{nonfunctional:network:headerbody} we discussed the need to validate
headers from upstream peers. In general, header validation requires information
from the ledger state. For example, in order to verify whether a Shelley header
was produced by the right node, we need to know the stake distribution (recall
that in Shelley the probability of being elected a leader is proportional to the
stake); this information is precisely what is captured by the
\lstinline!LedgerView! (\cref{consensus:class:ledgerview}). However, we cannot
update the ledger state with block headers only, we need the block bodies: after
all, to stay with the Shelley example, the stake evolves based on the
transactions that are made, which appear only in the block bodies.

Not all is lost, however. The stake distribution used by the Shelley ledger for
the sake of the leadership check \emph{is not the \emph{current} stake
distribution}, but the stake distribution as it was at a specific point in the
past. Moreover, that same stake distribution is then used for all leadership
checks in a given period of time.\footnote{The exact details of precisely
\emph{how} the chain is split is not relevant to the consensus layer, and is
determined by the ledger layer.} In the depiction below, the stake distribution
as it was at point $b$ is used for the leadership checks near the current tip,
the stake distribution at point $a$ was used before that, and so forth:
%
\begin{center}
\begin{tikzpicture}
%                             /--------\
%                             |        |
%                             *        v  tip
% 1 -----+------------+------------+-----+
%        |       *    |            |     * current stake
% 0 -----+------------+------------+-----|
%   -10  -8           -4           0     2
%                |          ^
%                \----------/
\draw (-10, 0.5) node{\ldots};
\draw (-10,  0) --  (2, 0);
\draw (-10,  1) --  (2, 1);
\draw  (-8,  0) -- (-8, 1);
\draw  (-4,  0) -- (-4, 1);
\draw   (0,  0) --  (0, 1);
\draw   (2,  0) --  (2, 1) node[above]{tip};
\draw  (-6, -0.2) node {$\underbrace{\hspace{3.8cm}}$};
\draw  (-2, -0.2) node {$\underbrace{\hspace{3.8cm}}$};
\draw  ( 2, -0.2) node {$\underbrace{\hspace{3.8cm}}$};
\draw [thick, arrows={-Triangle}] (-9, -1) node[fill=white] {$\ldots$}-- (-6, -1) -- (-6, -0.3);
\draw [thick, arrows={-Triangle}] (-5, 0.5) node[fill=white] {$\mathstrut a$} -- (-5, -1) -- (-2, -1) -- (-2, -0.3);
\draw [thick, arrows={-Triangle}] (-1, 0.5) node[fill=white] {$\mathstrut b$} -- (-1, -1) -- (2, -1) -- (2, -0.3);
\end{tikzpicture}
\end{center}
%
This makes it possible to \emph{forecast} what the stake distribution (i.e.,
the ledger view) will be at various points. For example, if the chain looks like
%
\begin{center}
\begin{tikzpicture}
\draw (-10,    0.5) node{\ldots};
\draw (-10,    0) -- (-0.5, 0);
\draw (-10,    1) -- (-0.5, 1);
\draw  (-8,    0) -- (-8,   1);
\draw  (-4,    0) -- (-4,   1);
\draw  (-0.5,  0) -- (-0.5, 1) node[above]{tip};
\draw  (-6,   -0.2) node {$\underbrace{\hspace{3.8cm}}$};
\draw  (-2,   -0.2) node {$\underbrace{\hspace{3.8cm}}$};
\draw  ( 2,   -0.2) node {$\underbrace{\hspace{3.8cm}}$};
\draw [thick, arrows={-Triangle}] (-9, -1) node[fill=white] {$\ldots$}-- (-6, -1) -- (-6, -0.3);
\draw [thick, arrows={-Triangle}] (-5, 0.5) node[fill=white] {$\mathstrut a$} -- (-5, -1) -- (-2, -1) -- (-2, -0.3);
\draw [thick, arrows={-Triangle}] (-1, 0.5) node[fill=white] {$\mathstrut b$} -- (-1, -1) -- (2, -1) -- (2, -0.3);
\draw (0, 0.5) node[left] {$\mathstrut c$};
\draw (0, 0.5) node[right] {$\mathstrut d$};
\draw (4, 0.5) node[right] {$\mathstrut e$};
\end{tikzpicture}
\end{center}
%
then we can ``forecast'' that the stake distribution at point $c$ will be the one
established at point $a$, whereas the stake distribution at point $d$ will be the
one established at point $b$. The stake distribution at point $e$ is however not
yet known; we say that $e$ is ``out of the forecast range''.

\subsection{Code}

Since we're always forecasting what the ledger would look like \emph{if it would
be advanced to a particular slot}, the result of forecasting is always something
ticked:\footnote{Actually we never deal with an \emph{unticked} ledger view.}
%
\begin{lstlisting}
data Forecast a = Forecast {
      forecastAt  :: WithOrigin SlotNo
    , forecastFor :: SlotNo -> Except OutsideForecastRange (Ticked a)
    }
\end{lstlisting}
%
Here \lstinline!forecastAt! is the tip of the ledger in which the forecast was
constructed and \lstinline!forecastFor! is constructing the forecast for a
particular slot, possibly returning an error message of that slot is out of
range. This terminology---a forecast constructed \emph{at} a slot
and computed \emph{for} a slot---is used throughout both this technical report
as well as the consensus layer code base.

\subsection{Ledger view}
\label{forecast:ledgerview}

For the ledger view specifically, the \lstinline!LedgerSupportsProtocol!
class (\cref{ledger:api:LedgerSupportsProtocol}) requires a function
%
\begin{lstlisting}
ledgerViewForecastAt ::
     LedgerConfig blk
  -> LedgerState blk
  -> Forecast (LedgerView (BlockProtocol blk))
\end{lstlisting}
%
This function must satisfy two important properties:
%
\begin{description}
\item[Sufficient range]

When we validate headers from an upstream node, the most recent usable ledger
state we have is the ledger state at the intersection of our chain and the chain
of the upstream node. That intersection will be at most $k$ blocks back, because
that is our maximum rollback (\cref{consensus:overview:k}). Furthermore, it is
only useful to track an upstream peer if we might want to adopt their blocks,
and we only switch to their chain if it is longer than ours
(\cref{consensus:overview:chainsel}). This means that in the worst case
scenario, with the intersection $k$ blocks back, we need to be able to evaluate
$k + 1$ headers in order to adopt the alternative chain. However, the range of a
forecast is based on \emph{slots}, not blocks; since not every slot may contain
a block (\cref{time:slots-vs-blocks}), the range needs to be sufficient to
\emph{guarantee} to contain at least $k + 1$ blocks\footnote{Due to a
misalignment between the consensus requirements and the Shelley specification,
this is not the case for Shelley, where the effective maximum rollback is in
fact $k - 1$; see \cref{shelley:forecasting}).}; we will come back to this in
\cref{future:block-vs-slot}.

The network layer may have additional reasons for wanting a long forecast
range; see \cref{nonfunctional:network:headerbody}.

\item[Relation to ticking]
Forecasting is not the only way that we can get a ledger view for a particular
slot; alternatively, we can also \emph{tick} the ledger state, and then ask
for the ledger view at that ticked ledger state. These two ways should give us
the same answer:
%
\begin{equation}
\begin{array}{lllll}
\mathrm{whenever} &
\mathtt{forecastFor} \; (\mathtt{ledgerViewForecastAt}_\mathit{cfg} \; l) \; s & = & \mathtt{Right} & l' \\
\mathrm{then} & \mathtt{protocolLedgerView}_\mathit{cfg} \; (\mathtt{applyChainTick}_\mathit{cfg} \; s \; l) & = && l'
\end{array}
\end{equation}
%
In other words, whenever the ledger view for a particular slot is within the
forecast range, then ticking the ledger state to that slot and asking for the
ledger view at the tip should give the same answer. Unlike forecasting, however,
ticking has no maximum range. The reason is the following fundamental difference between these two concepts:
%
\begin{quote}
\textbf{(Forecast vs. ticking)} When we \emph{forecast} a ledger view, we are
predicting what that ledger view will be, \emph{no matter which blocks will  be
applied to the chain} between the current tip and the slot of the forecast. By
contrast, when we \emph{tick} a ledger, we are applying any time-related
changes to the ledger state in order to apply the \emph{next} block; in other
words, when we tick to a particular slot, \emph{there \emph{are} no blocks in
between the current tip and the slot we're ticking to}. Since there are no
intervening blocks, there is no uncertainty, and hence no limited range.
\end{quote}
\end{description}

\section{Queries}
\label{ledger:queries}

\section{Abandoned approach: historical states}
