\chapter{System Architecture}

\section{Data Flow in a Node}
Nodes maintain connections with the peers that have been chosen
with help of the peer selection process.
Suppose node $A$ is connected to node $B$.
The Ouroboros protocol schedules a node $N$ to generate a new block in a given time slot.
Depending on the location of nodes $A$, $B$ and $N$ in the network topology and whether the new
block arrives first at $A$ or $B$, $A$ can be either up-stream or down-stream of $B$.
Therefore, node $A$ runs an instance of the client side of the chain-sync mini protocol
that talks with a server instance of chain-sync at node $B$ and also a server instance of chain sync
that talks with a client instance at $B$.
The situation is similar for the other mini protocols (block fetch, transaction submission, etc).
The set of mini protocols that runs over a connection is determined by the version of the network
protocol, i.e.  Node-to-Node, Node-to-Wallet and Node-to-Chain-Consumer
connections use different sets of mini protocols (e.g. different protocol
versions).  The version is negotiated when a new connection is established
using protocol which is described in Chapter~\ref{connection-management}.
\hide{Add description of this protocol in Chapter~\ref{connection-management}
and link it.}

\begin{figure}[ht]
  \pgfdeclareimage[height=15cm]{node-diagram-chains-state}{figure/node-diagram-concurrency.pdf}
  \begin{center}
    \pgfuseimage{node-diagram-chains-state}
  \end{center}
  \caption{Data flow inside a Node}
  \label{node-diagram-concurrency}
\end{figure}

Figure~\ref{node-diagram-concurrency} illustrates parts of the data flow in a node.
Circles represents a thread that runs one of the mini protocols (the mini protocols are explained in
Chapter~\ref{state-machine-section}).
There are two kinds of data flows:
mini protocols communicate with mini protocols of other nodes by sending and receiving messages;
and, within a node, they communicate by reading from- and writing to- a shared
mutable variable, which are represented by boxes in
Figure~\ref{node-diagram-concurrency}.
\href{https://en.wikipedia.org/wiki/Software_transactional_memory}{Software transactional memory}
(STM) is a mechanism for safe and lock-free concurrent
access to mutable state and the reference implementation makes intensive use of
this abstraction (see \cite{stm:harris2006}).

\section{Congestion Control}
A central design goal of the system is robust operation at high workloads.
For example, it is a normal working condition of the networking design
that transactions arrive at a higher rate than the number
that can be included in blockchain.
An increase of the rate at which transactions are submitted must not cause a decrease
of the block chain quality.

Point-to-point TCP bearers do not deal well with overloading.
A TCP connection has a certain maximal bandwidth,
i.e. a certain maximum load that it can handle relatively reliably under normal conditions.
If the connection is ever overloaded,
the performance characteristics will degrade rapidly unless
the load presented to the TCP connection is appropriately managed.

At the same time, the node itself has a limit on the rate at which it can process data.
In particular, a node may have to share its processing power with other processes that run on the
same machine/operation system instance, which means that a node may get slowed down for some reason,
and the system may get in a situation
where there is more data available from the network than the node can process.
The design must operate appropriately in this situation and recover form transient conditions.
In any condition, a node must not exceed its memory limits,
that is there must be defined limits, breaches of which being treated like protocol violations.

Of cause it makes no sense if the system design is robust,
but so defensive that it fails to meet performance goals.
An example would be a protocol that never transmits a message unless it has received an
explicit ACK for the previous message. This approach might avoid overloading the network,
but would waste most of the potential bandwidth.


\section{Real-time Constraints and Coordinated Universal Time}
Ouroboros models the passage of physical time as an infinite sequence of time slots,
i.e. contiguous, equal-length intervals of time,
and assigns slot leaders (nodes that are eligible to create a new block) to those time slots.
At the beginning of a time slot, the slot leader selects the block chain and transactions that are the basis
for the new block, then it creates the new block and sends the new block to its peers.
When the new block reaches the next block leader before the beginning of next time slot,
the next block leader can extend the block chain upon this block (if the block
did not arrive on time the next leader will create a new block anyway).

There are some trade-offs when choosing the slot time that is used for the protocol but
basically the slot length should be long enough such that a new block has a good chance to reach the
next slot leader in time.
A chosen value for the slot length is 20 seconds.
It is assumed that the clock skews between the local clocks of the nodes is small with respect to the
slot length.

However, no matter how accurate the local clocks of the nodes are with respect to the time slots
the effects of a possible clock skew must still be carefully considered.
For example, when a node time-stamps incoming blocks with its local clock time, it may encounter
blocks that are created in the future
with respect to the local clock of the node.
The node must then decide whether this is because of a clock skew or whether the node considers this
as adversarial behavior of an other node.

\wip{TODO :: get feedback from the researchers on this. Tentative policy: allow 200ms to 1s
explain the problem in detail.
A node cannot forward a block from the future.
This is complicated !
}
