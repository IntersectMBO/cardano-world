\chapter{Multiplexing mini-protocols}
\label{chapter:multiplexer}

\section{The Multiplexing Layer}
\label{multiplexing-section}
Multiplexing is used to run several mini protocols in parallel over
a bidirectional bearer (for example a TCP connection).
Figure~\ref{mux-diagram} illustrates multiplexing of three mini-protocols over
a single duplex bearer.  The multiplexer guarantees a fixed pairing of
mini-protocol instances, each mini-protocol only communicates with its counter
part on the remote end.

\begin{figure}[ht]
\pgfdeclareimage[height=7cm]{mux-diagram}{figure/mux.png}
\begin{center}
\pgfuseimage{mux-diagram}
\end{center}
\caption{Data flow though the multiplexer and de-multiplexer}
\label{mux-diagram}
\end{figure}


The multiplexer is agnostic to the bearer it runs over, however it assume that
the bearer guarantees an ordered and reliable transport layer\footnote{Slightly
more relaxed property is required: in order delivery of multiplexer segments
which belong to the same mini-protocol.} and it requires the bearer to be
\href{https://www.wikiwand.com/en/Duplex_(telecommunications)\#/Full-duplex}{full-duplex}
to allow simultaneous reads and writes\footnote{Note that one can always pair
two unidirectional bearers to form a duplex bearer, we use this to define
a duplex bearer out of unix pipes, or queues (for intra-process communication
only).}.  The multiplexer is agnostic to the serialisation used by
a mini-protocol (which we specify in section~\ref{chapter:mini-protocols}).
Multiplexer specifies its own framing / binary serialisation format, which is
described in section~\ref{section:wire-format}.  The multiplexer allows to use
each mini-protocol in either direction.

The multiplexer exposes interface which hides all the multiplexer details,
a single mini-protocol communication can be written as if it would only
communicate with its instance on the remote end.  When the muliplexer is
instructed to send bytes of some mini-protocol, it splits the data into
segments, adds a segment header, encodes it and transmits the segments over the
bearer.  When reading data from the network, segment's headers are used
reassemble mini-protocol byte streams.

\subsection{Wire Format}
\label{section:wire-format}

\begin{table}
  \begin{center}
    \begingroup
    \setlength{\tabcolsep}{3pt}
    \begin{tabular}{|c|c|c|c|c|c|c|c|c|c|c|c|c|c|c|c|c|c|c|c|c|c|c|c|c|c|c|c|c|c|c|c|}
      \hline
      0&1&2&3&4&5&6&7&8&9&0&1&2&3&4&5&6&7&8&9&0&1&2&3&4&5&6&7&8&9&0&1 \\ \hline
      \multicolumn{32}{|c|}{Transmission Time} \\ \hline
      \multicolumn{1}{|c|}{$M$}
      &\multicolumn{15}{|c|}{Mini Protocol ID}
      &\multicolumn{16}{|c|}{Payload-length $n$} \\ \hline
      \multicolumn{32}{|c|}{} \\
      \multicolumn{32}{|c|}{Payload of $n$ Bytes} \\
      \multicolumn{32}{|c|}{} \\ \hline
    \end{tabular}
    \endgroup
    \caption{Multiplexer's segment binary encoding, see
    \href{https://input-output-hk.github.io/ouroboros-network/network-mux/Network-Mux-Codec}{Network.Mux.Codec}.}
    \label{segment-header}
  \end{center}
\end{table}

Table~\ref{segment-header} shows the layout of the data segments of the multiplexing protocol
in big-endian bit order.  The segment header contains the following data:
\begin{description}
\item[Transmission Time]
  The transmission time is a time stamp based the lower 32 bits of the sender's monotonic clock with a
  resolution of one microsecond.
\item[Mini Protocol ID] The unique ID of the mini protocol as in
  tables~\ref{table:node-to-node-protocol-numbers}
    and~\ref{table:node-to-client-protocol-numbers}.
\item[Payload Length] The payload length is the size of the segment payload in Bytes.
  The maximum payload length that is supported by the multiplexing wire format is $2^{16}-1$.
  Note, that an instance of the protocol can choose a smaller limit for the size of segments it transmits.
\item[Mode] The single bit $M$ (the mode) is used to distinct the dual instances of a mini protocol.
  The mode is set to $0$ in segments from the initiator, i.e. the side that initially has agency and
  $1$ in segments from the responder.
\end{description}

\subsection{Fairness and Flow-Control in the Multiplexer}
The Shelley network protocol requires that the multiplexer uses a fair
scheduling of the mini protocols.  Haskell implementation of
multiplexer uses a round-robin-schedule of the mini protocols to choose the
next data segment to transmit.  If a mini protocol does not have new data
available when it is scheduled, it is skipped.  A mini-protocol can transmit at
most one segment of data every time it is scheduled and it will only be
rescheduled immediately if no other mini protocol is ready to send data.

From the point of view of the mini protocols, there is a one-message buffer between the egress of
the mini protocol and the ingress of the multiplexer.
The mini protocol will block when it sends a message and the buffer is full.

A concrete implementation of a multiplexer may use a variety of data structures and heuristics to
yield the overall best efficiency.
For example, although the multiplexing protocol itself is agnostic to the underlying structure of
the data, the multiplexer may try to avoid splitting small mini protocol messages into two segments.
The multiplexer may also try to merge multiple messages from one mini protocol into a
single segment.
Note that, the messages within a segment must all belong to the same mini-protocol.

\subsection{Flow-control and Buffering in the Demultiplexer}
\label{mux-flow-control}
The demultiplexer eagerly reads data from the bearer.
There is a fixed size buffer between the egress of the demultiplexer and the ingress of
the mini protocols.
Each mini protocol implements its own mechanism for flow control which guarantees that this buffer
never overflows (see Section~\ref{pipelining}.).
If the demultiplexer detects an overflow of the buffer, it means that the peer violated the
protocol and the MUX/DEMUX layer shuts down the connection to the peer.

\todo[inline]{Specify ingress buffer sizes for each mini-protocol}

\section{Node-to-node and node-to-client protocol numbers}
\noindent\haddockref{Network.Mux.Types}{network-mux/Network-Mux-Types}
\newline\haddockref{Ouroboros.Network.NodeToNode}{ouroboros-network/Ouroboros-Network-NodeToNode}
\newline\haddockref{Ouroboros.Network.NodeToClient}{ouroboros-network/Ouroboros-Network-NodeToClient}

Ouroboros network defines two protocols: \emph{node-to-node} and
\emph{node-to-client} protocols.  \emph{Node-to-node} is used for inter node
communication across the Internet, while \emph{node-to-client} is a inter
process communication, usedy by clients, e.g. a wallet, db-sync, etc.  Each of them consists of a bundle of mini-protocols (see chapter~\ref{chapter:mini-protocols})
consists of a bundle of mini-protocols.  The protocol numbers of both protocols
are specified in tables~\ref{table:node-to-node-protocol-numbers}
and~\ref{table:node-to-client-protocol-numbers}.
\begin{table}[ht]
  \begin{center}
    \begin{tabular}{l|c}
      mini-protocol                                                      & mini-protocol number \\\hline
      \hyperref[handshake-protocol]{Handshake}                           & 0  \\
      \hyperref[chain-sync-protocol]{Chain-Sync} \small{(instantiated to headers)} & 2  \\
      \hyperref[block-fetch-protocol]{Block-Fetch}                       & 3  \\
      \hyperref[tx-submission-protocol]{TxSubmission}                    & 4  \\
      \hyperref[keep-alive-protocol]{Keep-alive}                         & 8  \\
    \end{tabular}
  \end{center}
  \caption{Node-to-node protocol numbers}
  \label{table:node-to-node-protocol-numbers}
\end{table}
\begin{table}[ht]
  \begin{center}
    \begin{tabular}{l|c}
      mini-protocol                                                     & mini-protocol number \\\hline
      \hyperref[handshake-protocol]{Handshake}                          & 0 \\
      \hyperref[chain-sync-protocol]{Chain-Sync} \small{(instantiated to blocks)} & 5 \\
      \hyperref[local-tx-submission-protocol]{Local TxSubmission}       & 6 \\
      \hyperref[local-state-query-protocol]{Local State Query}          & 7 \\
    \end{tabular}
  \end{center}
  \caption{Node-to-client protocol numbers}
  \label{table:node-to-client-protocol-numbers}
\end{table}
