\section{Coin and Multi-Asset Bundles}
\label{sec:coin-ma}

In this chapter we introduce an algebraic structure, $\ValMonoid$,
which we define to be a partially ordered abelian group with several additional functions.
The group structure and additional functions represent the
features required in order to treat elements as \emph{assets} (or bundles thereof),
which includes the ability to do certain accounting operations on them (addition, etc.),
as well as some other operations (see below).
These operations are used to describe the ShelleyMA transaction processing rules
without fixing a concrete underlying $\ValMonoid$.

In order to give a concrete ledger specification, a specific type
endowed with $\ValMonoid$ structure must be used in its place.
Depending on the concrete type selected for this purpose,
we obtain distinct ledgers. In particular, we get

\begin{itemize}
  \item the Allegra ledger rules with $\Coin$, and
  \item the Mary ledger rules with $\Value$ (a type that represents a heterogeneous collection of assets via
  single element)
\end{itemize}

Below we give the definitions of all the functions that must be defined on
$\Coin$ and $\Value$ in order for them to have the structure of a $\ValMonoid$.
In Section \ref{sec:other-valmonoids} we give several other types which we can meaningfully
support the definition of the required functions (addition, size, etc.), including
an optimized representation that more accurately represents the ledger implementation
of the multi-asset type.
These types are convertible to and from the $\Value$ type, and appear in other
parts of the system as a multi-asset representation which is more suitable in particular
usecases.

\subsection{$\ValMonoid$}

Figure \ref{fig:ValMonoid} gives names and types of functions which
make up the $\ValMonoid$ structure. The functions $\fun{zero}$ and $(+)$ are
part of the abelian group structure, but we include them here to give the
complete $\ValMonoid$ structure. The partial order is given by $\fun{pointwise}~\leq$.

Note that we use overloaded notation
for partial order comparisons of $\ValMonoid$ elements elsewhere in the spec. That is,
we use the comparison operators $=,~<,~>,~\leq,~\geq$ in place the following pointwise comparisons, respectively

\[ \fun{pointwise}~(=),~\fun{pointwise}~(<),~\fun{pointwise}~(>),~\fun{pointwise}~(\leq),~\fun{pointwise}~(\geq) \]

%%
%% Figure ValMonoid and its Functions
%%
\begin{figure}[htb]
  \begin{align*}
      & \mathsf{ValType} ~\in~ \mathsf{Type}  \\
      & \text{A type}
      \nextdef
      %
      & \fun{zero} ~\in~ \mathsf{ValType} \\
      & \text{Additive identity of $\mathsf{ValType}$}
      \nextdef
      %
      & (+) ~\in~ \mathsf{ValType} \to \mathsf{ValType} \to \mathsf{ValType}\\
      & \text{Addition on $\mathsf{ValType}$}
      \nextdef
      %
      & (*) ~\in~ \mathsf{Integer} \to \mathsf{ValType} \to \mathsf{ValType}\\
      & \text{Scale $\mathsf{ValType}$ value by integral constant}
      \nextdef
      %
      & \fun{coin} ~\in~ \mathsf{ValType} \to \Coin\\
      & \text{Return the Ada contained inside the $\mathsf{ValType}$ element}
      \nextdef
      %
      & \fun{inject} ~\in~ \Coin \to \mathsf{ValType}\\
      & \text{Create a $\mathsf{ValType}$ element containing only this amount of Ada}
      \nextdef
      %
      & \fun{adaOnly} ~\in~ \mathsf{ValType} \to \Bool \\
      & \fun{adaOnly} =
        \begin{cases}
          \True & \fun{inject} \circ \fun{coin}~ = ~\fun{id}_{\mathsf{ValType}} \\
          \False & \text{otherwise}
        \end{cases} \\
      & \text{Check if a given $\mathsf{ValType}$ element contains only Ada}
      \nextdef
      %
      & \fun{size} ~\in~ \mathsf{ValType} \to \MemoryEstimate\\
      & \text{Return the size, in words, of a $\mathsf{ValType}$ element}
      \nextdef
      %
      & \fun{pointwise} ~\in~ (\Integer \to \Integer \to \Bool) \to (\mathsf{ValType} \to \mathsf{ValType} \to \Bool)\\
      & \text{Pointwise comparison of two $\mathsf{ValType}$ elements}
  \end{align*}
  \caption{The $\ValMonoid$ Structure}
  \label{fig:ValMonoid}
\end{figure}

\subsection{$\Coin$ in the Allegra era}

For the Allegra era, we require the $\ValMonoid$ asset representation to support
only a single asset, Ada, represented via $\Coin$. This section defines the
$\ValMonoid$ functions for the $\Coin$ type, see Figure
\ref{fig:coin}. Addition and multiplication are defined in the usual way
(as on $\Integer$).

When multi-asset support on the ledger is introduced, Ada ($\Coin$) will still be
the most common type of asset on the ledger, as the ledger rules enforce that
some quantity of it (specified via
the $\fun{coinsPerUTxOWord}$ protocol parameter) must
be contained in every UTxO on the ledger.
It is the only
type of asset that can be used for all non-UTxO ledger accounting, including deposits,
fees, rewards, treasury, and the proof of stake protocol. For this reason, not
all occurrences of $\Coin$ inside a transaction or in the ledger state can or
should be replaced with the abstract $\ValMonoid$.

For the Allegra version of the function $\fun{policies}$, which operates on $\Coin$ as the
$\ValMonoid$. No policies are associated with $\Coin$, since minting of Ada is not allowed.

%%
%% Figure Coin Functions
%%
\begin{figure}[htb]
  \begin{align*}
      & \fun{policies} ~=~ \Nothing
      \nextdef
      & \fun{zero} ~=~ 0
      \nextdef
      %
      & \fun{coin} ~=~ \fun{id}_{\Coin}
      \nextdef
      %
      & \fun{inject} ~=~ \fun{id}_{\Coin}
      \nextdef
      %
      & \fun{size}~\wcard ~=~ 0
      \nextdef
      %
      & \fun{pointwise} ~=~ \fun{id}_{(\Integer \to \Integer \to \Bool)}
  \end{align*}
  \caption{$\ValMonoid$ structure with $\Coin$}
  \label{fig:coin}
\end{figure}



\noindent \textbf{Relation between $\fun{inject}$ and $\fun{coin}$.}
When composed as

\[\fun{coin} \circ \fun{inject} = \fun{id}_{\Coin}\]

these functions give the identity on $\Coin$, but composed in the opposite order as

\[\fun{inject} \circ \fun{coin}\]

they return an element of $\ValMonoid$ containing only the Ada contained in the original $\ValMonoid$ element.

\subsection{Multi-assets and $\Value$ in the Mary Era}

Elements of $\Value$ represent heterogeneous collections of assets,
both user-defined, and Ada. The Mary era ledger fixes the underlying $\ValMonoid$ to
$\Value$ in order to introduce multi-asset support.
This section describes the type $\Value$, the operations required on
it, its relation to $\Coin$. It also as gives the definitions of the
$\ValMonoid$ functions for $\Value$.

Figure \ref{fig:defs:value} introduces the basic types that make up the $\Value$ type,
as well as defining the functions required to for $\Value$ to be a $\ValMonoid$.

\begin{figure*}[t!]
  \emph{Derived types}
  %
  \begin{equation*}
    \begin{array}{r@{~\in~}l@{\qquad=\qquad}lr}
      \var{aname} & ~\AssetName & \mathsf{ByteString} \\
      \var{pid} & \PolicyID & \ScriptHash \\
      \var{adaID} & ~\AdaIDType & \{~\mathsf{AdaID}~\} \\
      \var{aid} & \AssetID & \AdaIDType \uniondistinct (\PolicyID \times \AssetName) \\
      \var{quan} & \Quantity & \Z \\
      \var{v}, \var{w} & \Value & \AssetID \mapsto_0 \Quantity
    \end{array}
  \end{equation*}
  %
  \emph{$\ValMonoid$ and auxiliary functions for $\Value$}
  %
  \begin{align*}
    & \fun{policies} ~\in~ \Value \to \powerset{\PolicyID} \\
    & \fun{policies}~v ~=~ \{~\var{pid}~\vert~(\var{pid,~\wcard})~\in~\supp{~v}~\}
    \nextdef
    %
    & \fun{anames} ~\in~ \Value \to \powerset{\AssetName} \\
    & \fun{anames}~v ~=~ \{~\var{aname}~\vert~(\var{\wcard,~aname})~\in~\supp{~v}~\}
    \nextdef
    %
    & \fun{zero} ~=~ \Nothing
    \nextdef
    %
    & \fun{coin}~\var{v} ~=~ \var{v}~\mathsf{AdaID}
    \nextdef
    %
    & \fun{inject}~c  ~=~ \mathsf{AdaID}~\mapsto_0~\var{c}
    \nextdef
    %
    & \fun{size} ~~~ \text{see Section \ref{sec:value-size}}
  \end{align*}
  \caption{$\ValMonoid$ Function Definitions and Auxiliary Functions for Value}
  \label{fig:defs:value}
\end{figure*}

\begin{itemize}
  \item $\PolicyID$ identifies monetary policies. A policy ID $\var{pid}$ is associated with a script
    $s$ such that $\fun{hashScript}~s~=~pid$. When a transaction attempts to create or destroy assets
    that fall under the policy ID $\var{pid}$,
    $s$ verifies that the transaction
    respects the restrictions that are imposed by the monetary policy.
    See sections \ref{sec:transactions} and \ref{sec:utxo} for details.

  \item $\AssetName$ is a byte string used to distinguish different assets with the same $\PolicyID$.
    Each $aname$ identifies a particular kind of asset out of all the assets under the
    $\var{pid}$ policy (but not necessarily among assets under other policies).
    The maximum length of this
    byte string is 32 bytes (this is not explicitly enforced in this spec).

  \item $\AssetID$ is either $\mathsf{AdaID}$ or a pair of a policy ID and an asset name.
  It is a unique and permanent
  identifier of an asset. That is, there are is no mechanism to change it or
  any part of it for any assets.

  Mary MA assets are fungible with each other if and only if they have to the same $\AssetID$.
  The reason the unique identifier is a pair of two elements (except for the non-mintable Ada case) is to allow
  minting arbitrary collections of unique assets under a single policy.

  \item $\mathsf{AdaID}$ is a special asset ID for Ada, different than all other asset IDs.
  It is a term of the single-term type $\AdaIDType$.
  It does not include a policy, so instead, the validation outcome in the presence
  of Ada in the $\fun{mint}$ field of the transaction is specified in the UTXO
  ledger rule. The rule disallows the $\fun{mint}$ field to contain Ada.

  \item $\Quantity$ is an integer type that represents an amount of a specific $\AssetName$. We associate
    a $q\in\Quantity$ with a specific asset to track how much of that asset is contained in a given asset value.

  \item $\Value$ is the multi-asset type that is used to represent
    a collection of assets, including Ada. This type is a finitely supported map.

    If $\var{aid}$ is an $\AssetID$ and $v \in \Value$,
    the quantity of assets with that assed ID contained in $v$ is $v~\var{aid}$.
    Elements of $\Value$ are sometimes also referred to as
    \emph{asset bundles}.

  \item $\fun{inject}$ takes a $\Coin$ amount and returns a $\Value$ containing
  that amount of Ada, ie. assets associated with the $\mathsf{AdaID}$ asset ID.

  \item $\fun{coin}$ returns the amount of Ada contained in an element of $\Value$ (ie.
  the amount of assets associated with the $\mathsf{AdaID}$ asset ID).
\end{itemize}

\noindent \textbf{Special Ada representation.}
Although all assets are native on the Cardano ledger (ie. the accounting and
transfer logic for them is done directly by the ledger), Ada is still treated in a
special way by the ledger rules, and is the most common type of asset on the ledger.
It can
be used for special purposes (such as fees) for which other assets cannot be used.
The underlying consensus algorithm relies on Ada in a way that
cannot be extended to user-defined assets.
Ada can also neither be minted nor burned.

Note that in the $\Value$ definition above, we pick a special asset ID for Ada, that
is not part of the type which represents the asset IDs for all other assets.
Combining the asset name and policy for Ada gives a type-level guarantee that there is exactly
one kind of asset that is associated with it in any way, rather than
deriving this guarantee as an emergent property of minting rules.

Additionally, not giving Ada an actual policy ID
(that could have a hash-associated policy) eliminates the possibility
certain cryptographic attacks.
We sometimes refer to Ada as the primary or principal currency. Ada does not,
for the purposes of the Mary ledger spec, have a $\PolicyID$ or an $\AssetName$.

\subsection{Pointwise Value Operations and Partial Order}

We have a number of operations on $\Value$ left to define for it to have
$\ValMonoid$ structure. These are binary relations, addition, and
scalar multiplication. These are all done
pointwise in accordance with the usual definitions of these operations on
finitely supported maps, see Figure \ref{fig:pointwise}.

\begin{figure*}[t!]
  \begin{align*}
    & v~+~w =& \{~ aid~\mapsto_0 ~((v~\var{aid})~*~(w~\var{aid})) ~\vert~ \var{aid} \in (\fun{dom}~v)\cup(\fun{dom}~w) ~\}
    \nextdef
    %
    & q~*~v =& \{~ aid~\mapsto_0 ~(q~*~(v~\var{aid})) ~\vert~ \var{aid} \in \fun{dom}~v~\}
    \nextdef
    %
    & \fun{pointwise}~R~v~w =& \forall~\var{aid} ~\in ~(\fun{dom}~v~\cup~\fun{dom}~w),~R~(v~\var{aid})~(w~\var{aid})
  \end{align*}
  \caption{Pointwise operations on Value}
  \label{fig:pointwise}
\end{figure*}
